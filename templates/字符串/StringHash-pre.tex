卡常版本《author: sha7dow
字符串 Hash (卡常版本)(只能单/双哈希)
1-base
h[i] = s[1] * p^(i - 1) + s[2] * p^(i - 2) + ... + s[i];
要得到逆序哈希, 直接将字符串 reverse 后哈希即可.
注意不能有某个字符的哈希值为0!
传入的 n 为原字符串长度(即没有经过 s = " " + s 的长度), 传入的 s 为 1-base 数组.
代码中的哈希次数最好全部用 C 代替, 方便修改, 只需要修改 C = 即可(注意增加 p 和 mod)
Hash 结构体注意开在全局.


卡常方法
已经使用:
1. 用 int 代替 ll, 只在中间过程转成 ll.
2. 在保证不会错的情况下尽可能减少取模.
3. 预处理 pw 的过程完全没必要放在结构体中, 甚至没必要放在 solve 函数中, 直接放在多测(main函数)外即可.
4. 因为 array / pair 不能做为哈希表的键, 所以如果要用哈希表, 可以将 SY 改成 ll, ll 的前 30 位放一个哈希值, 
   后 30 位放一个(见卡常版本), 如果卡 unordered_map, 可以手写哈希表, 见 "整数哈希".

可以尝试:
1. 保证 p 大于最大字符哈希值 的前提下减小 p 和 mod.
2. 如果始终只需要一个 Hash 结构体, 就可以不用封装成结构体, 并且 vector 也可以换成数组, 同时删除 resize.
3. 实在不行试试 单哈希 或 自然溢出, 单哈希注意 mod 不能太小, 不然冲突概率太大了.
   单哈希需要自行修改一些地方, 包括删除 C, 二维数组改一维数组, ll 改 int ...(尽可能卡常) 