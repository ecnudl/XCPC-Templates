求环这里有两种方式和思路,第一种是用dfs求环,第二种是用并查集,二者相对来说并查集可能更为高效但是在代码难度方面用dfs会更加简单
这里以atcoder abc-contest311-C为例:
1. dfs求环:vis数组用来存储当前该节点是否判断过,ins数组是用来存储当前节点是否正在这个环中,一旦判断到环就可以直接exit,如果当前节点构成不了环,就再让它出队
2. 并查集求环,最开始做时困惑的点在于一个点的入度是可以大于1的,这样限制了回溯求环的思路。这时我们应该想到每个点的出度为1所以一次下手就可
3. 求最小环,用Floyd求最小环